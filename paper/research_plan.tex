\RequirePackage{plautopatch}
\RequirePackage[l2tabu, orthodox]{nag}

\documentclass[dvipdfmx]{jsarticle}			% for platex
% \documentclass[uplatex,dvipdfmx]{jlreq}		% for uplatex
\usepackage{graphicx}
%\title{レポートタイトル}

%\author{学生番号XXX-XXXX アカリク太郎}
%\date{\today}

% 行間調整
\usepackage{setspace}

% ページ番号を「今のページ数/総ページ数」(分数形式)で表示する
% https://zenn.dev/toru3/articles/34633b4c3d4682
\usepackage{fancyhdr,lastpage}
\fancypagestyle{mypagestyle}{
\lhead{} % ヘッダ左を空に
\rhead{} % ヘッダ右を空に
\cfoot{} % フッタ中央を空に
\rfoot{
    \thepage/\pageref{LastPage}} % フッタ右に"今のページ数/総ページ数"を設定
    \renewcommand{\headrulewidth}{0.0pt} % ヘッダの線を消す
}
\pagestyle{mypagestyle}

\begin{document}
%\maketitle
\begin{center}
{\normalsize 慶應義塾大学大学院政策・メディア研究科修士課程出願書類 研究計画書}\par
\vskip 0.75zh
{\Large 次世代インターネットに関する研究}\par
\vskip 0.75zh
{\normalsize
東洋大学 情報連携学部 \par
\texttt{\small s1f101800777@iniad.org}\par
\today\par
}
\end{center}
% abstract
\leftskip=2zw\rightskip=2zw
\centerline{\gtfamily 概要}\par
\smallskip
\begin{spacing}{0.8}
{\small 概要概要概要概要概要概要概要概要概要概要概要概要概要概要概要概要概要概要概要概要概要概要概要概要概要概要概要概要概要。}
\end{spacing}
\par

\section{Cloud LaTeXへようこそ}

Cloud LaTeXは,\LaTeX を使った文書の作成・管理をクラウド上で行えるWebサービスです.
\LaTeX を使うと,複雑な数式
\begin{equation}
\frac{\pi}{2} =
\left( \int_{0}^{\infty} \frac{\sin x}{\sqrt{x}} dx \right)^2 =
\sum_{k=0}^{\infty} \frac{(2k)!}{2^{2k}(k!)^2} \frac{1}{2k+1} =
\prod_{k=1}^{\infty} \frac{4k^2}{4k^2 - 1}
\end{equation}
を含んだ読みやすくきれいな文書作成ができます.

本サービスは,\LaTeX 文書をリアルタイムに保存・コンパイルし,ユーザーアカウント別に管理します.
そのため,本サービスにログインするだけで,どこからでも作業を再開でき,ファイルを持ち歩く必要はありません.
また,様々な \LaTeX テンプレートが用意されているので,手軽に文書を作り始めることができます.
\begin{figure}[ht]
\centering
\includegraphics[width=70mm]{figures/Sample.png}
\caption{ここにキャプションを挿入します}
\label{fig:model}
\end{figure}

Cloud LaTeXでは,作成されるPDFそのままのレイアウトで表示するPDFビューモードがあり,コンパイル画面を確認しながら文書を作成することができます(図\ref{fig:model})
日本語では,p\LaTeX / up\LaTeX / Lua\LaTeX でのコンパイルが可能です.
また,日本語や英語文書作成だけでなく,中国語・ハングルに対応した Xe\LaTeX のコンパイルも可能です.
ぜひ使ってみてください.
\end{document}